\documentclass[12pt]{exam}
\usepackage{amsmath,tikz,sagetex}
\usepackage[margin=2cm]{geometry}
\usepackage[hidelinks]{hyperref}


%%%%%%%%%%%%%%%%%%%%%%%%%%%%%%%%%%%%%
%      Change these variables       %
%%%%%%%%%%%%%%%%%%%%%%%%%%%%%%%%%%%%%

\newcommand{\examdate}{THE DATE}
\newcommand{\examnum}{EXAM NUM}
\newcommand{\course}{MATH NUM}
%for Draft watermark
%\newcommand{\isdraft}{x}
%for compressed spacing (i.e. vfill and newpage removal)
%\newcommand{\compresseddocument}{x}

%%%%%%%%%%%%%%%%%%%%%%%%%%%%%%%%%%%%%
%         Draft Watermark           %
%%%%%%%%%%%%%%%%%%%%%%%%%%%%%%%%%%%%%
\newcommand{\revisionlevel}{\ifdefined\isdraft
  \begin{tikzpicture}[remember picture,overlay]
  \node [rotate=60,scale=12,color=red!40,opacity=0.4] at (current page.center)
  {\textsf{DRAFT}};
\end{tikzpicture}
\else
  \relax
\fi}

%%%%%%%%%%%%%%%%%%%%%%%%%%%%%%%%%%%%%
%             Spacing               %
%%%%%%%%%%%%%%%%%%%%%%%%%%%%%%%%%%%%%
\newcommand{\compressspacing}[1]{\ifdefined\compresseddocument\relax\else#1\fi}
\newcommand{\elasticvfill}{\compressspacing{\vfill}}
\newcommand{\elasticnewpage}{\compressspacing{\newpage}}

\begin{document}

\pagestyle{headandfoot}
\header{\textbf{\course}}{\textbf{\examnum}}{\textbf{\examdate}\revisionlevel}
\firstpageheadrule
\runningheadrule
\firstpageheader{\textbf{\large \course}}{\textbf{\large \examnum}}{\textbf{\large \examdate}}
\firstpagefooter{\revisionlevel}{}{}
\runningfooter{\revisionlevel}{}{\thepage\ of \numpages}


\parindent 0ex

\begin{center}

\ifdefined\compresseddocument

{\large Student information goes here, if not compressing spacing.}

\rule[1ex]{\textwidth}{.1pt}

\else

\rule[1ex]{\textwidth}{.1pt}

\vspace{1cm}

\begin{tabular}{ll}
\sc{Name:} & \makebox[8cm]{\hrulefill} \\ \\ 
\sc{Instructor:} & \makebox[8cm]{\hrulefill} \\
\end{tabular}

\vspace{1cm}

\rule[1ex]{\textwidth}{.1pt}

\fi

\elasticvfill

\addpoints

\gradetable[v][questions]

\elasticvfill

\end{center}

\ifdefined\compresseddocument
\rule[1ex]{\textwidth}{.1pt}
\begin{center}
{\large Exam instructions go here, if not compressing spacing.}
\end{center}
\else
\LARGE
\begin{center}
\textbf{Instructions}
\end{center}
\normalsize

\rule[1ex]{\textwidth}{.1pt}

\noindent
\begin{enumerate}
\item This exam has $\numpages$ pages, including the cover sheet. There are $\numquestions$ problems, for a total of $\numpoints$ points. Please ensure no pages are missing.
\item You will have 50 minutes to complete this exam.  One may use an aproved calculator, but no notes or books are allowed.
\item Please make the steps you follow clear, and box your final answer.
\item When you are finished, please check your work carefully.
\end{enumerate}

\fi

\rule[1ex]{\textwidth}{.1pt} 

\elasticnewpage

\begin{questions}

\question[5]
Find all antiderivatives of the function $f(x) = \displaystyle\frac{e^{5x}}{3}-\frac{1}{x}$.

\elasticvfill

\question[5] 
Evaluate the indefinite integral $\displaystyle \int x^2 \sqrt[3]{x}dx$.

\elasticvfill

\elasticnewpage

\noaddpoints \question[10] \addpoints
Given $\displaystyle f(x,y) = x^2-4xy+10x-6y^2$
\begin{parts}
  \part[5] Find the point(s) where $f(x,y)$ has possible relative maima or minima.
  \elasticvfill
  \part[5] Using the second derivative test, classify the point(s) you find.
  \elasticvfill
\end{parts}

\elasticnewpage

\question[15] Find the area of the region bounded by $y = \frac{1}{x^2}$, $y=x$ and $y = \frac{x}{8}$ for $x\geq 0$. The following diagram may prove helpful:
\begin{sagesilent}
    a(x) =  1 / ( x * x )
    b(x) = x
    c(x) = x / 8
    p = plot(a, xmin = 0.00001, xmax = 3, ymin = 0, ymax = 3) + plot(b, xmin = 0, xmax = 3, color = 'red')+ plot(c, xmin = 0, xmax = 3, color = 'green')
    p.fontsize(25)
\end{sagesilent}

\begin{center}
\sageplot[scale=.3]{p}
\end{center}
However, be sure to justify all of your work, including where the functions intersect.

\elasticvfill

\end{questions}

\end{document}
